
\subsection{PokemonGo}

\begin{itemize}
\item{En los generadores (agJugador y agPknodo) se restringe el dominio bajo la condici\'on de que la coordenada exista en el mapa. Dado que las \'unicas coordenada validas son las del mapa, consideramos que no tiene sentido axiomatizar sobre coordenada fuera del mapa.}

\item{No se admiten jugadores distintos con el mismo nombre, lo que nos parecio razonable en este contexto.}

\item{Los jugadores que llegan a 5 sanciones son los eliminados, no son removidos del sistema pGo para controlar que no puedan volver a agregarse.}

\end{itemize}


\subsection{Pokenodo}

\begin{itemize}
\item{Representa al \'area que rodea al pokemon, donde se encuentran los jugadores en posici\'on de capturarlo. Es por esto que maneja la cuenta de movimientos necesarios para la captura.}
\end{itemize}

\subsection{Pokemon}

\begin{itemize}
\item{Debido a que el tad Pokenodo y el tad PokemonGo se encargan del comportamiento referente a captura, cuenta de movimientos e \'indice de rareza, para representar al Pokemon nos basta con usar un renombre del tad String.}
\end{itemize}

\subsection{Jugador}

\begin{itemize}
\item{Las operaciones sobre jugadores se restringen en su mayor\'ia a jugadores conectados, por el enunciado del TP.}
\end{itemize}

