
\subsection{PokemonGo}

\begin{itemize}
\item{En los generadores (agJugador y agPknodo) se restringe el dominio bajo la condici\'on de que la coordenada exista en el mapa. Dado que las unicas coordenada validas son las del mapa, concideramos que no tiene sentido axiomatizar sobre coordenada fuera del mapa.}

\item{No se admiten jugadores distintos con el mismo nombre, lo que es razonable en este contexto.}

\item{Los jugadores que llegan a 5 sanciones permanecen, no son borrados de la instancia para evitar volver a agregarlos.}

\end{itemize}


\subsection{Pokenodo}

\begin{itemize}
\item{Reprezenta al \'area que rodea al pokemon, donde se encuentran los jugadores en posicion de capturar a ese pokemon. Es por eso que maneja la cuenta de movimientos necesarios para la captura.}
\end{itemize}

\subsection{Pokemon}

\begin{itemize}
\item{Debido a que el tad Pokenodo y el tad PokemonGo se encargan del comportamiento referente a captura, cuenta de movimientos e \'indice de rareza, para representar al pokemon basta con usar un renombre del tad String.  }
\end{itemize}

\subsection{Jugador}

\begin{itemize}
\item{Las operaciones sobre jugadores se restringen en su mayoria a jugadores conectados, por el enunciado del TP.}
\end{itemize}

